\subsection{Elettroforesi SDS-PAGE seguito da Western Blot}
L’elettroforesi SDS-PAGE seguita dal Western blot è una tecnica combinata che consente di separare le proteine in base alla loro dimensione e successivamente identificarle in modo specifico con anticorpi. Questa esperienza mostra come visualizzare la presenza e la quantità di proteine target in un campione complesso, fornendo dati chiave per studi di espressione e regolazione proteica.

\subsection{Cromatografia di affinità}
La cromatografia di affinità è un metodo di purificazione proteica altamente selettivo, basato sull’interazione specifica tra la proteina di interesse e una matrice legata a ligandi specifici. Questa esperienza descrive come sfruttare queste interazioni per isolare una proteina target da un lisato cellulare, ottenendo un campione altamente purificato per studi successivi.

\subsection{ICC (Immunocitochimica)}
L’immunocitochimica (ICC) è una tecnica che permette di localizzare e visualizzare proteine specifiche direttamente all’interno delle cellule. Sfruttando anticorpi fluorescenti o enzimatici, questa esperienza mostra come ottenere un’immagine precisa della distribuzione intracellulare delle proteine, fornendo informazioni fondamentali per lo studio delle funzioni cellulari.
