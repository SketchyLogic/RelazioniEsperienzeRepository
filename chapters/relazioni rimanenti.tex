\subsection{Mantenimento di colture cellulari di mammifero}
Il mantenimento di colture cellulari di mammifero è una tecnica di laboratorio fondamentale per studiare la biologia cellulare, esprimere proteine ricombinanti o analizzare risposte cellulari. Questa esperienza si concentra su come preparare, nutrire e mantenere linee cellulari in condizioni ottimali, garantendo la vitalità e la riproducibilità degli esperimenti successivi.

\subsection{Estrazione di proteine da cellule di mammifero}
L’estrazione proteica da cellule di mammifero è una procedura che consente di isolare proteine totali o frazionate dalle cellule in coltura. Questa esperienza illustra come rompere le membrane cellulari in modo controllato, preservando l’integrità delle proteine e preparando lisati cellulari per analisi successive come Western blot o saggi enzimatici.

\subsection{Elettroforesi SDS-PAGE seguito da Western Blot}
L’elettroforesi SDS-PAGE seguita dal Western blot è una tecnica combinata che consente di separare le proteine in base alla loro dimensione e successivamente identificarle in modo specifico con anticorpi. Questa esperienza mostra come visualizzare la presenza e la quantità di proteine target in un campione complesso, fornendo dati chiave per studi di espressione e regolazione proteica.

\subsection{Cromatografia di affinità}
La cromatografia di affinità è un metodo di purificazione proteica altamente selettivo, basato sull’interazione specifica tra la proteina di interesse e una matrice legata a ligandi specifici. Questa esperienza descrive come sfruttare queste interazioni per isolare una proteina target da un lisato cellulare, ottenendo un campione altamente purificato per studi successivi.

\subsection{ICC (Immunocitochimica)}
L’immunocitochimica (ICC) è una tecnica che permette di localizzare e visualizzare proteine specifiche direttamente all’interno delle cellule. Sfruttando anticorpi fluorescenti o enzimatici, questa esperienza mostra come ottenere un’immagine precisa della distribuzione intracellulare delle proteine, fornendo informazioni fondamentali per lo studio delle funzioni cellulari.
