\normalsize
\section {Estrazione di proteine da cellule di mammifero}

\subsection{Introduzione}
L’estrazione delle proteine da cellule di mammifero è un passaggio fondamentale per lo studio dell’espressione e della regolazione proteica.
In questa esperienza, partendo dai pellet cellulari ottenuti dall’esperienza precedente, le cellule \texttt{HEK293T} vengono lisate per ottenere una frazione di proteine citosoliche.
Il lisato proteico sarà poi utilizzato per preparare campioni destinati alla successiva analisi tramite \textbf{Western blot} (SDS-PAGE seguito da immunoblotting).
A differenza di un’estrazione totale delle proteine (che includerebbe anche le frazioni nucleari e di membrana), questa procedura si focalizza sulla frazione citosolica.
Un aspetto importante dell’esperienza è l’uso di un \textbf{buffer di lisi arricchito con inibitori di proteasi e fosfatasi}, per preservare l’integrità delle proteine e la loro fosforilazione durante l’estrazione.

\begin{insightBox}
	\textbf{Insight:} La differenza tra un’estrazione “citoplasmatica” e una “totale” delle proteine sta nella forza del detergente usato e nelle condizioni del buffer.
	In questa esperienza si usa un \textbf{buffer delicato con Triton} che rompe solo la membrana plasmatica, rilasciando le \textbf{proteine citosoliche}.
	Le strutture intracellulari (nucleo, mitocondri, reticolo endoplasmatico) restano intatte, conservando le rispettive proteine al loro interno.
	Se invece volessimo fare un’estrazione “totale”, servirebbero detergenti più forti (come SDS o NP-40) o l’uso di ultrasuoni per rompere anche questi compartimenti.
\end{insightBox}

\begin{percheBox}
	\textbf{Perché è importante inibire le fosfatasi:}
	Le fosfatasi rimuovono i gruppi fosfato dalle proteine, alterando il loro stato di fosforilazione.
	Poiché la fosforilazione regola l’attività, la stabilità e le interazioni delle proteine, impedirne la rimozione durante l’estrazione è essenziale per \textbf{preservare la forma funzionale e fisiologica delle proteine} come avviene in vivo.
\end{percheBox}


\subsection{Obiettivo}
L’obiettivo dell’esperienza è ottenere lisati proteici citosolici dalle cellule \texttt{HEK293T}, mantenendo la loro integrità e preparare i campioni per le successive analisi di Western blot.

\subsection{Strumentazione}
\begin{itemize}
	\item Cappa a flusso laminare
	\item Micropipette e puntali sterili
	\item Vortex
	\item Centrifuga refrigerata (4~$^\circ$C)
	\item Ghiaccio
\end{itemize}

\subsection{Soluzioni e reagenti}\footnotesize{
  \begin{itemize}
	\item \textbf{TRITON buffer}: buffer di lisi per rompere la membrana cellulare e liberare le proteine.
	\item \textbf{PP inhibitors}: mix di inibitori di proteasi e fosfatasi per prevenire la degradazione delle proteine durante l’estrazione.
	\item \textbf{LLS 5X} (\textit{Laemmli Loading Sample Buffer} 5X): tampone di caricamento per la corsa elettroforetica.
	\item \textbf{Pellet cellulari (campioni e controllo negativo)} costituiti da:
	\begin{itemize}\footnotesize
		\item pellets di cellule trasfettate con \texttt{pcDNA3-$\beta$Catenin-HA} (βCat-HA)
		\item controllo negativo (CTRL neg) con \texttt{pcDNA3} vuoto
	\end{itemize}

\end{itemize}
}


\begin{percheBox}
	\textbf{Perché si usa il Laemmli Loading Sample Buffer (LLS 5X):}

	Questo tampone di caricamento è essenziale per la preparazione dei campioni proteici destinati alla corsa su gel SDS-PAGE (vedi esperienze successive).
	Denatura le proteine, rompendo le strutture tridimensionali e uniformando la carica negativa grazie all’SDS.
	Il glicerolo aumenta la densità del campione, facilitando il caricamento nei pozzetti del gel.
	Infine, il bromofenolo blu permette di monitorare visivamente la corsa elettroforetica.
\end{percheBox}
\newpage

\begin{criticitaBox}
  \textbf{Criticità:} È fondamentale mantenere il campione di cellule lisate a temperature prossime o sotto lo zero durante l’intero processo, perché la bassa temperatura rallenta l’azione delle proteasi e delle fosfatasi cellulari.  
  Questi enzimi, se attivi, potrebbero degradare le proteine o rimuovere modifiche post-traduzionali importanti, come la fosforilazione, compromettendo così la qualità delle analisi successive.
\end{criticitaBox}
\vspace{1em}
\subsection{Procedura}


\begin{enumerate}\footnotesize
\item \textbf{Preparazione dei campioni}
\begin{itemize}
\item Recuperare 4 campioni contenenti i pellet cellulari:
\begin{itemize}
\item 1 tubo contenente cellule di controllo negativo (campione trasfettato con plasmide vuoto).
\item 3 tubi contenenti cellule \texttt{HEK293T} trasfettate con \texttt{pcDNA3-$\beta$Catenin-HA} (campioni 2-4).
\end{itemize}
\end{itemize}

\item \textbf{Preparazione del buffer di lisi TRITON}
\begin{itemize}
\item Aggiungere 6~$\mu$L della soluzione di inibitori PP (mix di inibitori di proteasi e fosfatasi) in un tubo contenente 1~mL di TRITON buffer.
\end{itemize}

\item \textbf{Lisi delle cellule}
\begin{itemize}
\item Aggiungere 60~$\mu$L del TRITON buffer arricchito con inibitori in ciascun tubo contenente i pellet cellulari.
\item Risospendere accuratamente i pellet usando una pipetta P200, evitando la formazione di bolle.
\item \textit{(Accorgimento: Per evitare bolle, settare la pipetta a 50~$\mu$L e operare lentamente!)}
\item Vortexare per pochi secondi, poi lasciare i tubi in ghiaccio per 10 minuti.
\item Ripetere il passaggio di vortex e ghiaccio per garantire una completa lisi delle cellule.
\end{itemize}

\item \textbf{Centrifugazione}
\begin{itemize}
\item Centrifugare i campioni alla massima velocità per 20 minuti a 4~$^\circ$C.
\item Dopo la centrifugazione, i pellet con le frazioni insolubili (membrane, organelli) verranno scartati.
\end{itemize}

\item \textbf{Raccolta del surnatante (lisato citosolico)}
\begin{itemize}
\item Prelevare con attenzione il surnatante contenente le proteine citosoliche e trasferirlo in nuovi tubi Eppendorf etichettati come segue:
\begin{itemize}
\item \texttt{CTRL neg} (Controllo negativo).
\item \texttt{$\beta$Cat-HA} (campioni 2-4).
\end{itemize}
\item Questi saranno i \textbf{tubi per la raccolta del lisato proteico}.
\end{itemize}

\item \textbf{Preparazione del campione per la corsa elettroforetica (Western blot)}
\begin{itemize}
\item Etichettare altri 4 tubi Eppendorf (\texttt{CTRL neg Wb} e \texttt{$\beta$Cat-HA Wb} per i campioni 2-4).
\item In ciascun tubo, aggiungere 20~$\mu$L del lisato proteico (prelevato dal passo precedente).
\item Aggiungere 5~$\mu$L di \texttt{LLS 5X buffer} (\textit{Laemmli Loading Sample Buffer 5X}) in ciascun tubo.
\item Mescolare bene per ottenere un volume finale di 25~$\mu$L.
\item Centrifugare per 30 secondi alla massima velocità.
\item Questi campioni sono pronti per essere analizzati con la tecnica Western Blot
\end{itemize}

\item \textbf{Conservazione}
\begin{itemize}
\item Conservare i campioni a -20~$^\circ$C fino al momento dell’uso.
\end{itemize}
\end{enumerate}

\newpage