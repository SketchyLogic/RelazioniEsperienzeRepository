\section {Mantenimento di colture cellulari di mammifero}

\subsection{Introduzione}
Il mantenimento di colture cellulari di mammifero è un passaggio fondamentale per numerosi studi di biologia molecolare e cellulare, come l’analisi dell’espressione proteica, la trasfezione di plasmidi e la caratterizzazione funzionale delle proteine. Questa esperienza si concentra sul mantenimento e la subcultura di cellule \texttt{HEK293T} trasfettate, ponendo attenzione a tutte le precauzioni per evitare contaminazioni e garantire l’integrità e la vitalità delle colture.

\subsection{Cellule HEK293T e relazione con T-antigene SV40}

L’SV40 (\textit{Simian Virus 40}) è un virus che infetta cellule di primate e il suo genoma contiene un gene che codifica per una proteina virale chiamata \textbf{T-antigene SV40}.  
Questa proteina ha la funzione di riconoscere e attivare l’\textbf{origine di replicazione SV40}, una sequenza di DNA specifica presente nel genoma del virus stesso.

Nelle applicazioni di biologia molecolare, l’origine di replicazione SV40 viene inserita in plasmidi per consentire la loro replicazione efficiente nelle cellule eucariotiche.  

Le cellule \texttt{HEK293T}, utilizzate in questa esperienza, sono state modificate per esprimere il T-antigene SV40 in modo stabile. Questo le rende capaci di riconoscere e attivare l’origine di replicazione SV40 presente in plasmidi \texttt{pcDNA.3-HA} e \texttt{pcDNA.3-$\beta$Catenin-HA}, facilitando così la replicazione e la stabilità del plasmide all’interno delle cellule.
\vspace{0.25em}\\
In sintesi, il T-antigene SV40 nelle cellule \texttt{HEK293T} consente una replicazione più rapida e abbondante dei plasmidi contenenti l’origine SV40, rendendole ideali per studi di espressione proteica e altri esperimenti in cui è necessario avere molte copie di un plasmide.

\subsection{Cos'è un plasmide taggato e a cosa serve}

I plasmidi utilizzati in questa esperienza, come \texttt{pcDNA3-HA} e \texttt{pcDNA3-$\beta$Catenin-HA}, sono \textbf{taggati} con una piccola sequenza aggiuntiva chiamata \textit{HA tag} (Hemagglutinin tag).  
Il \textbf{tag} è un breve peptide che viene fuso alla proteina di interesse e che non ne altera significativamente la funzione o la struttura.  
\vspace{0.5em}\\
\textbf{Come funziona il tag:}
\begin{itemize}\footnotesize
  \item Il tag \textit{HA} è riconosciuto da anticorpi specifici.
  \item Permette di \textbf{rilevare e quantificare} la proteina espressa durante gli esperimenti (es. Western blot, immunoprecipitazione).
  \item Facilita l’analisi della localizzazione intracellulare in tecniche come l’immunocitochimica.
  \item Garantisce che le proteine esogene (ad esempio, la $\beta$-Catenina) possano essere \textbf{identificate con facilità} rispetto alle proteine endogene normalmente presenti nelle cellule.
\end{itemize}

In sintesi, i tag come \textit{HA} sono strumenti per confermare la presenza e la corretta espressione di una proteina d'interesse della cellula host.

\newpage
\subsection{Obiettivo}
L’obiettivo di questa esperienza è mantenere e suddividere correttamente le colture cellulari di \texttt{HEK293T} in adesione, garantendo la vitalità e la purezza delle cellule per futuri esperimenti di espressione proteica e analisi molecolare.
\vspace{0.5em}\\
In particolare, l’esperienza prevede la produzione e la gestione di due popolazioni di cellule:
\begin{itemize}\footnotesize
  \item Una coltura trasfettata con \texttt{pcDNA3-$\beta$Catenin-HA} per studiare gli effetti dell’espressione di questa proteina nelle cellule.
  \item Una coltura trasfettata con \texttt{pcDNA3-HA} (\textbf{plasmide vuoto}) come controllo negativo.
\end{itemize}
Questa distinzione permette di confrontare le due condizioni sperimentali e ottenere risultati affidabili e riproducibili.
molecolare.\\
\vspace{0.5em}
\subsection{Strumenti e soluzioni}
\twoColumnLayout
  {Strumentazione:}
  {
  \item Cappa a flusso laminare
  \item Microscopio invertito
  \item Pipette e puntali sterili
  \item Tubi Falcon sterili (15 mL e 50 mL)
  \item Pipetta aspirante
  \item Centrifuga da laboratorio
  \item Incubatore a 37~$^\circ$C con 5\% CO$_2$
  \item Etanolo al 70\% per la sterilizzazione
  }
  {Soluzioni e reagenti:}
  {
  \item Cellule \texttt{HEK293T} trasfettate con\\ \texttt{pcDNA.3-HA} o \texttt{pcDNA.3-$\beta$Catenin-HA}
  \item Terreno di crescita completo\\ (DMEM + 10\% FBS + Pen/Strep 1X)
  \item PBS 1X (senza Ca$^{2+}$ e Mg$^{2+}$)
  \item Soluzione di tripsina-EDTA
  }
\vspace{1em}
\begin{insightBox}
  \textbf{Insight:} Il \textbf{microscopio invertito} è progettato con la sorgente luminosa e gli obiettivi posti sotto il piano di osservazione. Questo consente di osservare comodamente cellule in coltura in piastre o contenitori in plastica, dove le cellule aderiscono al fondo. Rispetto ai microscopi convenzionali, facilita la visione di cellule vive senza la necessità di campioni sottili.
\end{insightBox}
\begin{percheBox}
  \textbf{Perché 5\% CO$_2$?}

  L’incubatore a 5\% CO$_2$ ricrea le condizioni fisiologiche tipiche dell’ambiente cellulare. Il CO$_2$ si dissolve nel terreno di coltura, contribuendo a mantenere il \textbf{pH stabile e ottimale} (circa 7.4). Senza questa atmosfera controllata, il pH tenderebbe a salire o scendere, compromettendo la vitalità e la funzionalità delle cellule in coltura.
\end{percheBox}

\begin{insightBox}
  \textbf{Insight:} Ogni componente del mantenimento delle cellule ha un ruolo fondamentale:
  \begin{itemize}\footnotesize
    \item \textbf{Terreno di crescita completo (DMEM + 10\% FBS + Pen/Strep 1X)}: fornisce i nutrienti (glucosio, amminoacidi, sali) e i fattori di crescita (FBS) essenziali per la sopravvivenza e la proliferazione cellulare. La Penicillina/Streptomicina previene contaminazioni batteriche.
    \item \textbf{PBS 1X (senza Ca$^{2+}$ e Mg$^{2+}$)}: tampone isotonico usato per lavare le cellule e rimuovere residui di terreno o sieri. L’assenza di ioni Ca$^{2+}$ e Mg$^{2+}$ facilita il distacco cellulare perché questi ioni stabilizzano le giunzioni cellulari.
    \item \textbf{Soluzione di tripsina-EDTA}: enzima proteolitico che rompe le connessioni tra le cellule e la matrice di adesione, consentendo di staccarle e passare a un nuovo contenitore (splitting).
  \end{itemize}
\end{insightBox}

\begin{insightBox}
  \textbf{Insight:} La tripsina non è selettiva per le proteine di adesione, ma l’aggiunta dell’EDTA le rende i primi bersagli dell’enzima.\\
  L’EDTA chela (lega) gli ioni Ca$^{2+}$ e Mg$^{2+}$, essenziali per la stabilità delle proteine di adesione (come integrine e cadherine). Senza questi ioni, le proteine perdono la loro integrità strutturale e si indeboliscono.\\
  Questo le rende \textbf{più vulnerabili} all’azione proteolitica della tripsina, che quindi le taglia per prime, facilitando il distacco delle cellule dal substrato.
\end{insightBox}


\newpage

\subsection{Procedura: Preparazioni}

\begin{enumerate}\footnotesize
  \item Etichettare due provette Falcon da 15\,mL con “DMEM completo” e “HEK293T-βCatenina-HA”.
  \item Disinfettare l’area della cappa a flusso laminare con etanolo al 70\% e siglare i tubi necessari.
  \item Osservare le cellule al microscopio invertito per controllare la morfologia e la confluenza.
  \item Se la confluenza è superiore all’80-90\%, procedere con lo splitting.
\end{enumerate}

\begin{percheBox}
\footnotesize{  \textbf{Perché:} Quando la confluenza supera l’80-90\%, le cellule iniziano a competere per spazio e nutrienti, il che può portare a stress cellulare, alterazioni morfologiche e perdita di vitalità. Effettuare lo splitting in questa fase mantiene le cellule in uno stato ottimale per la crescita e previene fenomeni di contatto-inibizione che ridurrebbero la loro capacità di proliferare.}
\end{percheBox}

\subsection{Procedura: Preparazione terreno di coltura completo (5mL)}
\begin{enumerate}\footnotesize
  \item Utilizzando una pipetta sterile, prelevare:
  \begin{itemize}
    \item 4.45mL di DMEM.
    \item 0.5mL (500µL) di siero fetale bovino (FBS) per una concentrazione finale del 10\%.
    \item 50µL di soluzione Pen/Strep 100X per una concentrazione finale 1X.
  \end{itemize}
  \item Aggiungere nell’ordine: prima DMEM, poi FBS e infine Pen/Strep.
  \item Miscelare delicatamente ruotando la provetta per garantire l’omogeneità della soluzione.
  \item Tenere il terreno di coltura a 37$^\circ$C o utilizzarlo subito per la semina delle cellule.
\end{enumerate}

\subsection{Procedura}
\begin{enumerate}\footnotesize
  \item Fase I: Splitting
  \begin{itemize}
    \item Rimuovere il terreno di crescita con una pipetta sterile.
    \item Aggiungere 3 mL di PBS 1X e lavare delicatamente le cellule per 10 secondi.
    \item Aggiungere 1 mL di tripsina-EDTA e incubare 6 minuti a temperatura ambiente.
    \item Osservare la distaccazione delle cellule al microscopio.
    \item Aggiungere 4 mL di DMEM completo per neutralizzare la tripsina.
    \item Risospendere con pipetta sierologia le cellule per singolarizzarle (volume totale 5ml: 1ml + 4ml DMEM) per evitare la formazione di aggregati cellulari.
    \item Calcolare e prelevare il volume necessario per una diluizione 1:10, lasciando questo volume nella piastra.
    \item Trasferire il resto della sospensione cellulare in un tubo Falcon da 15 mL.
    \item Aggiungere 5 mL di DMEM completo alla piastra e mescolare delicatamente.
    \item Incubare le piastre a 37~$^\circ$C con 5\% CO$_2$.
  \end{itemize}
  \item Procedere con la raccolta delle cellule (Fase II):
  \begin{itemize}
    \item Etichettare 3 tubi Eppendorf e trasferirvi 1.3 mL di sospensione cellulare per tubo.
    \item Centrifugare a 4~$^\circ$C, 1000 rpm per 5 minuti.
    \item Rimuovere il surnatante e risospendere il pellet in 1 mL di PBS 1X.
    \item Ripetere centrifugazione e lavaggio altre due volte.
    \item Rimuovere accuratamente il surnatante e mantenere i pellet in ghiaccio.
    \item Conservare i pellet a -80~$^\circ$C per successive estrazioni proteiche.
  \end{itemize}
\end{enumerate}

\begin{percheBox}
\footnotesize{  \textbf{Perché:} Il \textbf{DMEM completo} neutralizza l’azione della tripsina grazie alla presenza del siero fetale bovino (FBS), che contiene inibitori naturali delle proteasi (ad esempio $\alpha$-1-antitripsina). Questi inibitori bloccano l’attività proteolitica della tripsina, proteggendo le cellule e prevenendo un’eccessiva digestione delle loro proteine di membrana.}
\end{percheBox}

\begin{insightBox}
\footnotesize{  \textbf{Insight:} Dopo la raccolta e la conservazione a -80$^\circ$C, le cellule non sono più vitali e non possono proliferare. Tuttavia, il loro contenuto (proteine, RNA, DNA) rimane integro e può essere recuperato per analisi biochimiche e molecolari.}
\end{insightBox}

\subsection{Conclusioni}
Questa procedura garantisce la vitalità e la purezza delle cellule \textit{HEK293T} trasfettate, preparando il materiale biologico necessario per futuri esperimenti di lisi cellulare e analisi proteica. Il rispetto delle condizioni sterili e dei passaggi di lavaggio e splitting è fondamentale per mantenere l’affidabilità delle colture e prevenire contaminazioni.

\newpage
