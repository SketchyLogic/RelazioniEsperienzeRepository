\section {Miniprep - Preparazione di DNA plasmidico tramite lisi alcalina}

\subsection{Introduzione - Contesto}


\subsection{Obiettivo}
L’esperimento mira a estrarre il DNA plasmidico in forma pura, separandolo da componenti cellulari come proteine, RNA e DNA genomico al fine di ottenere un campione adatto a successive analisi molecolari.

\subsection{Strumentazione}
\begin{itemize}
	\item Micropipette e puntali sterili
	\item Provette Eppendorf (1.5-2 ml)
	\item Centrifuga da banco
	\item Vortex
	\item Congelatore o ghiaccio secco
\end{itemize}

\subsection{Soluzioni e reagenti}
\begin{itemize}
	\item {Soluzione I}: per risospendere il pellet batterico (spesso contiene Tris, EDTA e glucosio)
	\item {Soluzione II}: per la lisi cellulare (NaOH e SDS, potente detergente)
	\item {Soluzione III}: tampone di neutralizzazione (acetato di potassio, a pH acido)
	\item {Etanolo 100\% o isopropanolo}: per precipitare il DNA
	\item {Etanolo 70\%}: per lavaggi finali
	\item {Tampone TE (Tris-EDTA)}: per la risospensione del DNA plasmidico
	\item {RNAasi (opzionale)}: per degradare eventuale RNA contaminante
\end{itemize}

\subsection{Procedura}
\begin{enumerate}
    \item \textbf{Step}
    
    {\footnotesize \textbf{Obiettivo}: L'obiettivo di questo step}

    \begin{itemize}
        \item Centrifuga per 30 secondi a 12.000 rpm
        \item Centrifuga per 30 secondi a 12.000 rpm
        \item Capovolgi il tubo ed elimina il supernatante.
    \end{itemize}

    \begin{accorgimentoBox}
    \textbf{Accorgimento:}Box Accorgimento
    \end{accorgimentoBox}

    \begin{percheBox}
    \textbf{Perché:} Box Perche
    \end{percheBox}

    \begin{criticitaBox}
    \textbf{Criticità:} Box Criticita
    \end{criticitaBox}

    \item \textbf{Step}
    \item \textbf{Step}
    \item \textbf{Step}
    \item \textbf{Step}
\end{enumerate}
