\section {Restrizione del DNA ed analisi elettroforetica}

\subsection{Introduzione}

\subsubsection{Tecniche e tecnologie utilizzate}
\begin{itemize}
  \item Selezione di una porzione di DNA con enzimi di restrizione
  
  {\footnotesize La restrizione del DNA è una tecnica fondamentale della biologia molecolare che sfrutta enzimi chiamati \textit{enzimi di restrizione} per tagliare il DNA in punti specifici. Questi enzimi riconoscono e clivano sequenze particolari, producendo frammenti di dimensioni definite. In questo esperimento utilizzeremo l’enzima di restrizione EcoRI, che taglia il DNA plasmidico pUC18 in corrispondenza dei suoi siti di riconoscimento. L’uso di un controllo negativo (campione non trattato) ci permetterà di confrontare la digestione completa con il DNA intatto.}
  
  \item Gel elettroforesi (per un controllo qualitativo)
  
  {\footnotesize L’elettroforesi su gel di agarosio è una tecnica che permette di separare le molecole di DNA in base alle loro dimensioni e al loro stato conformazionale. In un campo elettrico, le molecole di DNA cariche negativamente migrano verso il polo positivo. Tuttavia, la migrazione non dipende solo dalla lunghezza del frammento, ma anche dallo stato topologico: le molecole superavvolte (supercoiled), aperte circolarmente o lineari hanno velocità di migrazione differenti, anche a parità di lunghezza. Questo è particolarmente importante per analizzare plasmidi e frammenti di restrizione.}
\end{itemize}

\subsubsection{Informazioni generali sull'enzima EcoRI}

EcoRI è un enzima di restrizione di tipo II, estratto da \textit{Escherichia coli}, che taglia il DNA a doppio filamento in corrispondenza di una sequenza palindromica specifica: 5'–GAATTC–3'. Il taglio avviene tra la guanina e la adenina, generando estremità coesive (sticky ends) con un breve tratto di singolo filamento. L’azione di EcoRI è altamente specifica e dipende dalle condizioni ottimali di temperatura e cofattori ionici (ad esempio Mg$^{2+}$).


\subsection{Obiettivo}
Lo scopo dell’esperienza è la di digestione di DNA plasmidico pUC18 con l’enzima EcoRI poi verificare l’efficacia della taglio mediante elettroforesi su gel di agarosio confrontanto il prodotto della reazione con pUC18 originale (controllo negativo).
\newpage

\subsection{Strumentazione e reagenti (digestione enzimatica)}

\textbf{Strumentazione:}
\begin{itemize}
  \item Micropipette e puntali
  \item Eppendorf
  \item Beuta
  \item Falcon
  \item Vortex
\end{itemize}

\textbf{Soluzioni e reagenti:}
\begin{itemize}
  \item enzima di restrizione EcoRI
  \item Buffer specifico per EcoRI
  \item DNA plasmidico pUC18
\end{itemize}

\subsection{Strumentazione e reagenti (gel elettroforesi)}

\textbf{Strumentazione:}
\begin{itemize}
  \item Vaschette elettroforetiche
  \item Transilluminatore\footnote{Dispositivo che emette luce UV o blu per visualizzare i frammenti di DNA nel gel dopo la colorazione.}
  \item Microonde
\end{itemize}

\textbf{Soluzioni e reagenti:}
\begin{itemize}
  \item TAE 50X\footnote{Tampone concentrato (Tris-Acetato-EDTA) da diluire per l’elettroforesi. Mantiene stabile il pH e garantisce la conduzione elettrica durante la corsa del gel.}
  \item SyberSafe\footnote{Colorante fluorescente che si lega al DNA e RNA, consentendo la visualizzazione dei frammenti nel gel sotto luce UV o blu. È più sicuro e meno tossico rispetto al bromuro d’etidio.}
  \item Gel d’agarosio
  \item DNA ladder\footnote{Mix di frammenti di DNA di dimensioni note, usato come marcatore per confrontare e stimare la lunghezza dei frammenti di DNA separati nel gel.}
\end{itemize}


\subsection{Procedura per digestione del DNA}
\begin{enumerate}
  \begin{noSplitBlock}
\item \textbf{Preparazione della miscela di digestione}

      {\footnotesize \textbf{Obiettivo}: Preparare la reazione di digestione con l’enzima EcoRI per tagliare il DNA plasmidico pUC18.}

      \begin{itemize}
          \item 10~$\mu$l di pUC18
          \item 1~$\mu$l di EcoRI (enzima di restrizione)
          \item 2.5~$\mu$l di buffer 10X
          \item 11.5~$\mu$l di H$_2$O
      \end{itemize}

      \begin{criticitaBox}
          \textbf{Criticità:} L’acqua va aggiunta per prima e l’enzima per ultimo per ridurre al minimo il rischio di denaturazione dell’enzima e garantire la sua attività.
      \end{criticitaBox}

      \begin{percheBox}
          \textbf{Perché:} L’enzima EcoRI taglia specificamente le sequenze riconosciute, frammentando il DNA plasmidico in siti precisi.
      \end{percheBox}
\end{noSplitBlock}


\begin{noSplitBlock}
\item \textbf{Preparazione del controllo negativo (senza enzima)}

      {\footnotesize \textbf{Obiettivo}: Preparare un controllo negativo per verificare la specificità dell’azione dell’enzima EcoRI.}

      \begin{itemize}
          \item 10~$\mu$l di pUC18
          \item - (nessun enzima)
          \item 2.5~$\mu$l di buffer 10X
          \item 12.5~$\mu$l di H$_2$O
      \end{itemize}

      \begin{percheBox}
          \textbf{Perché:} Il controllo negativo permetterà di verificare che eventuali modifiche al DNA plasmidico siano dovute solo all’azione dell’enzima EcoRI e non ad altri fattori.
      \end{percheBox}
\end{noSplitBlock}

\begin{noSplitBlock}
\item \textbf{Incubazione delle reazioni}

      {\footnotesize \textbf{Obiettivo}: Favorire l’azione dell’enzima EcoRI sul DNA plasmidico e garantire la massima efficienza di taglio.}

      \begin{itemize}
          \item Incubare entrambe le provette (digestione e controllo) a 37~$^\circ$C per 1-2 ore.
      \end{itemize}

      \begin{accorgimentoBox}
          \textbf{Accorgimento:} Evitare sbalzi di temperatura o mescolamenti eccessivi durante l’incubazione, per non danneggiare l’enzima o il DNA.
      \end{accorgimentoBox}

      \begin{percheBox}
          \textbf{Perché:} La temperatura di 37~$^\circ$C è ottimale per l’attività dell’enzima EcoRI, assicurando un taglio efficiente e specifico del DNA plasmidico.
      \end{percheBox}
\end{noSplitBlock}

\end{enumerate}

\subsection{Procedura preparazione Gel Elettroforesi}
\begin{enumerate}

\begin{noSplitBlock}
\item \textbf{Preparazione della soluzione di agarosio}

      {\footnotesize \textbf{Obiettivo}: Preparare un gel di agarosio allo 0.8\% per l’analisi elettroforetica dei campioni.}

      \begin{itemize}
          \item Pesare 0.6~g di agarosio e aggiungere 1.6~ml di TAE 50X.
          \item Aggiungere acqua distillata fino a 80~ml.
          \item Scaldare la miscela a microonde (in una beuta) senza far bollire, mescolando di tanto in tanto per sciogliere completamente l’agarosio.
      \end{itemize}

      \begin{accorgimentoBox}
          \textbf{Accorgimento:} Sciogliere bene l’agarosio, evitando la formazione di grumi o la fuoriuscita del liquido.
      \end{accorgimentoBox}
\end{noSplitBlock}

\begin{noSplitBlock}
\item \textbf{Aggiunta del colorante}
      \begin{itemize}
          \item Lasciare raffreddare leggermente la soluzione di agarosio.
          \item Aggiungere 4~$\mu$l di SyberSafe, mescolare accuratamente.
      \end{itemize}

      \begin{percheBox}
          \textbf{Perché:} Il SyberSafe lega gli acidi nucleici e consente la loro visualizzazione sotto luce UV o LED.
      \end{percheBox}

      \begin{criticitaBox}
        \textbf{Criticità:} Il SyberSafe deve essere aggiunto quando la soluzione di agarosio è ancora fluida ma non eccessivamente calda, poiché temperature troppo alte potrebbero compromettere la stabilità del colorante, riducendo l’efficienza di legame al DNA e la fluorescenza per la successiva analisi.
      \end{criticitaBox}
\end{noSplitBlock}

\begin{noSplitBlock}
\item \textbf{Colatura del gel}

      {\footnotesize \textbf{Obiettivo}: Preparare un gel uniforme e senza bolle pronto per la separazione elettroforetica.}

      \begin{itemize}
          \item Versare la soluzione di agarosio nella vaschetta di corsa, senza pettinino.
          \item Lasciare solidificare il gel a temperatura ambiente.
          \item Quando il gel è solido o quasi, aggiungere il pettinino per formare i pozzetti
          \item Coprire il gel TAE 1x
      \end{itemize}

      \begin{accorgimentoBox}
        \textbf{Accorgimento:} Colare il gel senza il pettinino per evitare la formazione di bolle intorno ai denti e assicurare pozzetti regolari e ben formati. Inserire il pettinino solo dopo che la soluzione di agarosio si è raffreddata leggermente o si è completamente solidificata.
      \end{accorgimentoBox}

      \begin{percheBox}
\textbf{Perché:} Il TAE “dentro” al gel serve a stabilizzare la struttura del gel e mantenere un ambiente idoneo durante la formazione del gel stesso, mentre quello “fluido” nella vaschetta serve a garantire la conduzione elettrica e a stabilizzare il pH durante la corsa elettroforetica.
\end{percheBox}


\end{noSplitBlock}

\end{enumerate}

\begin{noSplitBlock}
  \subsection{Preparazione della corsa elettroforetica}

      {\footnotesize \textbf{Obiettivo}: Caricare i campioni nei pozzetti, alimentare la vaschetta, ottenere una buona separazione dei vari frammenti di acidi nucleici.}

      \begin{itemize}
          \item Preparare i campioni:
          \begin{itemize}
              \item 25~$\mu$l di pUC18 digerito + 5~$\mu$l di Sample Buffer
              \item 25~$\mu$l di pUC18 non digerito + 5~$\mu$l di Sample Buffer
              \item 25~$\mu$l di RNA totale + 5~$\mu$l di Sample Buffer (Vedi esperienza "Estrazione RNA Totale con TRIzol")
              \item 5~$\mu$l di DNA ladder (marcatore di pesi molecolari)
          \end{itemize}
          \item Caricare i campioni nei pozzetti.
          \item Cablare l'alimentatore alla vaschetta ed impostare il voltaggio dell’alimentatore tra 90 e 100~V.
          \item Lasciare correre per il tempo necessario a ottenere una buona separazione.
      \end{itemize}

    \begin{accorgimentoBox}
      \textbf{Accorgimento:} Se la corsa elettroforetica viene lasciata procedere troppo a lungo, i frammenti di DNA o RNA più piccoli potrebbero migrare fuori dal gel, mentre quelli più grandi potrebbero risultare distorti o sovrapposti. Questo comprometterebbe la qualità e l’interpretazione dei risultati.
    \end{accorgimentoBox}
    \end{noSplitBlock}

    \begin{noSplitBlock}
  \subsection{Osservare il gel nel transilluminatore}

      {\footnotesize \textbf{Obiettivo}: Osservare e documentare la separazione dei frammenti di DNA o RNA nel gel dopo l’elettroforesi.}

      \begin{itemize}
          \item Posizionare il gel sul transilluminatore (UV o LED).
          \item Accendere la lampada e regolare la luminosità per ottenere la migliore visibilità dei frammenti.
          \item Osservare e fotografare i pattern di bande ottenute, confrontandoli con il DNA ladder caricato.
      \end{itemize}

      \begin{criticitaBox}
      \textbf{Criticità:} Non osservare direttamente la luce UV, utilizzare sempre la copertura protettiva del transilluminatore per evitare danni agli occhi.
      \end{criticitaBox}
\end{noSplitBlock}










\subsection{Come interpretare l’analisi}

Come interpretare la migrazione dei frammenti? 
\newline
Vi è una relazione tra [dimensione, forma] degli acidi nucleici e velocità nel gel di agarosio.

\begin{itemize}
  \item \textbf{Dimensione degli acidi nucleici}: i frammenti più piccoli migrano più velocemente nel gel e si trovano più lontano dal pozzetto di caricamento. I frammenti più grandi, invece, si muovono più lentamente e si localizzano più vicino all'origine.
  
  \item \textbf{Topologia del DNA}: frammenti aventi la stessa lunghezza possono migrare in modo diverso a seconda della loro struttura topologica:
  \begin{itemize}
    \item \textit{Forme supercoiled} (superavvolte) migrano più velocemente perché sono più compatte.
    \item \textit{Forme lineari} migrano a una velocità intermedia.
    \item \textit{Forme nicked o circolari rilassate} (aperti) migrano più lentamente poiché occupano più spazio nella matrice del gel.
  \end{itemize}
  
  \item \textbf{Confronto con il ladder}: la scala di riferimento (DNA ladder) aiuta a stimare le dimensioni dei frammenti campione, confrontandone la distanza di migrazione.
\end{itemize}
\newpage