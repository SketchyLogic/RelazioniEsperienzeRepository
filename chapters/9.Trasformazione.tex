\section {Preparazione piastre per trasformazione DH5$\alpha$}

\subsection{Introduzione}
Le piastre di LB-agar arricchite con antibiotici e indicatori cromogenici sono fondamentali per la selezione e il riconoscimento dei batteri trasformati con plasmidi contenenti geni di resistenza o reporter come \textit{lacZ}. In questa sezione viene spiegato come preparare due tipologie di piastre: \textbf{Piastra con Kan/Caf} e \textbf{Piastra con Amp/IPTG/Xgal}.

\subsection{Approfondimento su Piastra con Kan/Caf}
La \textbf{Piastra con Kan/Caf} contiene \textbf{kanamicina} e \textbf{cloramfenicolo} come antibiotici selettivi. Questi antibiotici inibiscono la crescita dei batteri che non contengono il plasmide con i geni di resistenza corrispondenti. Le colonie che crescono su queste piastre possiedono un plasmide che conferisce resistenza a entrambi gli antibiotici. Questa strategia è utile per la selezione di cloni trasformati in esperimenti di clonaggio o in espressioni in cui è richiesta la selezione multipla.

\subsection{Approfondimento su Piastra con Amp/IPTG/Xgal}
La \textbf{Piastra con Amp/IPTG/Xgal} combina la selezione antibiotica (ampicillina) con un sistema cromogenico. 
\begin{itemize}\footnotesize
  \item \textbf{Ampicillina}: seleziona i batteri trasformati contenenti il plasmide con il gene di resistenza.
  \item \textbf{IPTG}: è un induttore del promotore \textit{lac}, stimolando l’espressione del gene \textit{lacZ} codificante la $\beta$-galattosidasi.
  \item \textbf{X-gal}: substrato cromogenico che, in presenza di $\beta$-galattosidasi, produce un composto blu.
\end{itemize}
Le colonie che esprimono $\beta$-galattosidasi appaiono blu, mentre quelle che contengono un inserto nel gene \textit{lacZ} (interruzione) restano bianche.

\subsection{Approfondimento tecnica blue-white screening}

La tecnica del \textbf{blue-white screening} è un metodo rapido ed efficace per identificare colonie batteriche che contengono plasmidi ricombinanti con l’inserto desiderato. Essa sfrutta il gene \textit{lacZ$\alpha$}, che codifica la subunità $\alpha$ della $\beta$-galattosidasi. Questo gene è presente nel plasmide e complementa una porzione mancante (\textit{lacZ$\omega$}) nel cromosoma del batterio ospite, ripristinando così l’attività dell’enzima.

\vspace{0.5em}

\textbf{Dopo la trasformazione}, alcuni batteri integrano il plasmide, mentre altri no.
\begin{itemize}\footnotesize
\item I batteri che non hanno integrato alcun plasmide vengono eliminati perché non possiedono il gene di resistenza agli antibiotici e quindi non sopravvivono sulle piastre contenenti antibiotico.
\item I batteri che hanno integrato il plasmide possiedono il gene di resistenza e quindi formano colonie.
\end{itemize}

\vspace{0.5em}

Tuttavia, c’è un’ulteriore distinzione: alcuni plasmidi integrati potrebbero essere “vuoti” (senza inserto), mentre altri contengono l’inserto del gene di interesse. È qui che entra in gioco il concetto di \textbf{“insertional inactivation”}.

\vspace{0.5em}

\textbf{Insertional inactivation} significa che l’inserto del gene di interesse viene clonato all’interno del gene \textit{lacZ$\alpha$}.
\begin{itemize}\footnotesize
\item \textbf{Plasmide vuoto (senza inserto):} il gene \textit{lacZ$\alpha$} è intatto e funziona. L’enzima $\beta$-galattosidasi viene prodotto e scinde il substrato cromogenico \textbf{X-gal}, generando un composto blu $\Rightarrow$ \textbf{colonie blu}.
\item \textbf{Plasmide con inserto:} l’inserto interrompe il gene \textit{lacZ$\alpha$}, inattivandolo. L’enzima $\beta$-galattosidasi non viene prodotto e l’X-gal non viene scisso $\Rightarrow$ \textbf{colonie bianche}.
\end{itemize}

\vspace{0.5em}

Questa distinzione visiva permette di selezionare rapidamente le colonie “bianche” (con inserto) e distinguere i cloni ricombinanti corretti, risparmiando tempo e risorse nella verifica del clonaggio.
La combinazione di antibiotico (per selezione positiva) e sistema cromogenico (per la distinzione tra plasmide vuoto e ricombinante) è una tecnica potente e versatile.

\newpage
\subsection{Obiettivo}
Preparare piastre LB-agar supplementate con antibiotici e substrati cromogenici per la successiva trasformazione e selezione dei batteri DH5$\alpha$.

  \vspace{1em}
\twoColumnLayout
    {Reagenti:}
    {
      \item Bilancia analitica
      \item Becher
      \item Bacchetta di vetro o magnete per agitazione
      \item Microonde
      \item Falcon
      \item Piastre Petri
      \item Pipette e puntali
      \item Cappa a flusso laminare
    }
  {Strumentazione:}
  {
  \item LB-agar solido
  \item Soluzioni stock di antibiotici:
    \begin{itemize}\footnotesize
      \item Kanamicina 1000X
      \item Cloramfenicolo (CAF) 1000X
      \item Ampicillina 1000X
    \end{itemize}
  \item Soluzioni di:
    \begin{itemize}\footnotesize
      \item IPTG 0.5 mM
      \item X-gal 80 $\mu$g/mL
    \end{itemize}
  \item Acqua milliQ
  }

\subsection{Procedura}
\begin{enumerate}\footnotesize
  \item Sciogliere il terreno LB-agar (se già solidificato) nel microonde o su piastra riscaldante.
  \item Lasciare raffreddare leggermente e, sotto cappa, trasferire 25 mL in una Falcon sterile.
  \item Aggiungere 25 $\mu$L di CAF 1000X e 25 $\mu$L di Kan 1000X. Mescolare e versare in una piastra Petri (\textbf{Piastra con Kan/Caf}).
  \item Ai rimanenti 25 mL aggiungere:
    \begin{itemize}
      \item 25 $\mu$L di Ampicillina (dallo stock 1000X)
      \item 25 $\mu$L di IPTG 0.5 mM
      \item 25 $\mu$L di X-gal 80 $\mu$g/mL
    \end{itemize}
    Mescolare bene e versare in un’altra piastra Petri (\textbf{Piastra con Amp/IPTG/Xgal}).
  \item Lasciare le piastre semi-aperte sotto cappa fino a completa solidificazione dell’agar.
  \item Chiudere e conservare le piastre a 4~$^\circ$C.
\end{enumerate}

\subsection{Conclusioni}
Le piastre preparate con antibiotici e substrati cromogenici sono pronte per la selezione dei batteri trasformati. La \textbf{Piastra con Kan/Caf} permette la selezione di plasmidi con geni di resistenza alla kanamicina o cloramfenicolo, mentre la \textbf{Piastra con Amp/IPTG/Xgal} consente sia la selezione con ampicillina sia la rivelazione di colonie “blue/white” per il gene \textit{lacZ}. Questa preparazione garantisce un ambiente selettivo e cromogenico essenziale per esperimenti di clonaggio e trasformazione.

\newpage
