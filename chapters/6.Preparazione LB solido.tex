\section{Preparazione di terreno LB solido}
\subsection{Introduzione}

La preparazione del terreno LB solido è un passaggio fondamentale per la crescita di \textit{Escherichia coli} su piastre. Il terreno LB (Luria-Bertani) fornisce una fonte ricca di nutrienti e supporta la crescita di colonie batteriche isolate.

\subsection{Composizione del terreno LB solido}

\begin{itemize}\footnotesize
    \item 1\% tryptone: miscuglio di peptidi e amminoacidi derivato dalla digestione parziale della caseina (una proteina del latte)
    \item 0.5\% estratto di lievito (yeast extract): fornisce vitamine, nucleotidi e altri fattori di crescita essenziali.
    \item 0.5\% NaCl: mantiene l’equilibrio osmotico e la stabilità ionica della soluzione (previene plasmolisi e lisi osmotica).
    \item 1.5\% agar batteriologico: solidificante derivato da alghe, consente la formazione di piastre solide per la crescita delle colonie.
\end{itemize}

\begin{notaBox}
  \textbf{Nota:} Le percentuali indicate (es. 1\% tryptone) rappresentano concentrazioni in peso/volume (w/v): ad esempio, 1\% w/v equivale a 1 g di triptone disciolto in 100 mL di acqua distillata. Non sono già soluzioni, ma polveri che vanno disciolte manualmente nella preparazione del terreno.
\end{notaBox}

\subsection{Obiettivo}

Lo scopo di questa esperienza è preparare 50 mL di terreno LB solido sterile per l’uso in piastre Petri e la successiva crescita di batteri.

\subsection{Strumentazione e reagenti}

\twoColumnLayout
    {Reagenti:}
    {
    \item Tryptone
    \item Estratto di lievito
    \item NaCl
    \item Agar batteriologico
    \item Acqua distillata
    }
  {Strumentazione:}
  {
    \item Bilancia analitica
    \item Becher
    \item Bacchetta di vetro o magnete per agitazione
    \item Bottiglia in vetro (o Pirex) adatta per autoclave
    \item Autoclave
  }

\subsection{Procedura}

\begin{enumerate}
        \item Pesare le quantità necessarie per 50 mL di terreno LB solido:
    \begin{itemize}
        \item 0.5 g di tryptone
        \item 0.25 g di estratto di lievito
        \item 0.25 g di NaCl
        \item 0.75 g di agar batteriologico
    \end{itemize}
    \item Aggiungere i componenti in un becher e mescolare con 50 mL di acqua distillata fino a completa dissoluzione.
    \item Trasferire la soluzione in una bottiglia adatta per l’autoclave.
    \item Attaccare un pezzetto di nastro da autoclave alla bottiglia per indicare l’avvenuta sterilizzazione.
    \item Sterilizzare la soluzione in autoclave a 121~$^\circ$C per 15-20 minuti.
    \item Lasciare raffreddare leggermente la soluzione prima dell’uso o versare nelle piastre Petri in condizioni sterili.
    
\end{enumerate}