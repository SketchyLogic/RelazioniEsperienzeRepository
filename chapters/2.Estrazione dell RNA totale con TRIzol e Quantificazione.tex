\section {Estrazione dell RNA totale con TRIzol e Quantificazione}

\subsection{Introduzione - Contesto}
La manipolazione dell’RNA richiede una particolare attenzione, poiché si tratta di una molecola estremamente instabile e soggetta a degradazione da parte delle RNasi. Per questo motivo, le procedure di estrazione dell’RNA devono essere svolte rapidamente, in condizioni controllate e utilizzando reagenti capaci di inattivare le nucleasi. Il metodo TRIzol\footnote{Reagente a base di fenolo e guanidina isotiocianato che consente di isolare RNA, DNA e proteine in un singolo passaggio.} rappresenta una soluzione efficace per ottenere RNA totale\footnote{Insieme di tutte le molecole di RNA presenti in una cellula o in un tessuto, comprese mRNA, tRNA, rRNA e RNA non codificanti.} da campioni cellulari, poiché combina la lisi cellulare, la denaturazione delle proteine e la separazione delle fasi in un unico passaggio, garantendo un’alta resa e una buona qualità dell’RNA.


\subsection{Obiettivo}
Ottenere un estratto di RNA totale, privo di contaminazioni proteiche o genomiche, che possa essere utilizzato per successive esperienze.

\subsection{Strumentazione}
\begin{itemize}
  \item Mortaio e pestello
  \item Micropipette e puntali
  \item Eppendorf
  \item Centrifuga
  \item Spettrofotometro a cuvetta
  \item Spettrofotometro nanodrop
\end{itemize}

\subsection{Soluzioni e reagenti}
\begin{itemize}
  \item TRIzol (Fenolo + Guanidina isocianato)
  \item Azoto liquido
  \item Cloroformio
  \item 2-propanolo
  \item Etanolo 70\%
  \item DEPC (DiEthyl PyroCarbonate)
\end{itemize}

\newpage

\subsection{Procedura}
\begin{enumerate}

\begin{noSplitBlock}
\item \textbf{Polverizzare le foglie con azoto liquido}

{\footnotesize \textbf{Obiettivo}: Ridurre in polvere le foglie in polvere e trasferile in eppendorf}

\begin{itemize}
  \item Raffredda il mortaio con l’azoto liquido, lasciando evaporare l’azoto in eccesso.
  \item Aggiungi i pezzi di foglia e tritali usando il pestello.
  \item La polvere ottenuta va trasferita in due provette Eppendorf da 2~ml, utilizzando una spatola pre-raffreddata.
\end{itemize}

\begin{percheBox}
\textbf{Perché:} L’azoto liquido congela rapidamente le foglie, impedendo che diventino pastose e facilitando la frantumazione in una polvere fine.
\end{percheBox}

\begin{accorgimentoBox}
\textbf{Accorgimento:} Usa una spatola pre-raffreddata per trasferire la polvere. In caso contrario, la polvere potrebbe attaccarsi alla spatola a causa dell’elevato cambio di temperatura.
\end{accorgimentoBox}

\begin{criticitaBox}
\textbf{Criticità:} Non aspettare troppo a lungo dopo la frantumazione, poiché l’RNA può iniziare a degradarsi a temperatura ambiente. Lavora in modo rapido e in un ambiente freddo.
\end{criticitaBox}
\end{noSplitBlock}

\begin{noSplitBlock}
\item \textbf{Aggiungere TRIzol (1000~$\mu$l ogni 50mg di tessuto) e agitare per 30 secondi.}
\end{noSplitBlock}

\begin{noSplitBlock}
\item \textbf{Attendere 5min}

{\footnotesize \textbf{Obiettivo}: Durante questo tempo, la miscela appare come una sospensione omogenea e torbida, senza fasi separate. Questo passaggio consente alle molecole di RNA di liberarsi dalle proteine e di rimanere in soluzione.}

\begin{percheBox}
\textbf{Perché:} Questo riposo facilita la completa dissociazione dei complessi nucleoproteici, favorendo l’isolamento selettivo dell’RNA nella fase acquosa successiva.
\end{percheBox}
\end{noSplitBlock}

\begin{noSplitBlock}
\item \textbf{Aggiungere 0.2~ml di cloroformio poi vortexare per 30s.}
\begin{itemize}
  \item Aggiungere 0.2~ml di cloroformio (1:5 con il TRIzol).
  \item Vortexare per 30s
\end{itemize}

\begin{percheBox}
\textbf{Perché:} Il cloroformio, un solvente organico non polare, induce la separazione della miscela in tre fasi ben distinte: la fase organica (inferiore) che contiene i lipidi e altre molecole idrofobe, la fase intermedia con i detriti cellulari e le proteine denaturate, e la fase acquosa (superiore), dove si concentrano le molecole idrofile come l’RNA.
\end{percheBox}

\begin{percheBox}
\textbf{Perché} il DNA non è nella stessa fase dell'RNA?
I frammenti di RNA sono 3 o 4 ordini di gradezza piu corti di quelli di DNA.

L’RNA è più solubile in acqua e ha gruppi OH più esposti.

Il DNA, più lungo e complesso, tende a precipitare o restare nella fase intermedia o organica
\end{percheBox}


\begin{criticitaBox}
\textbf{Criticità:} Lavorare sotto cappa in quanto il cloroformio è tossico.
\end{criticitaBox}
\end{noSplitBlock}

\begin{noSplitBlock}
\item \textbf{Attendere e poi centrifugare}

{\footnotesize \textbf{Obiettivo}: Ottenere una seprazione netta delle tre fasi, in particolare una visibile fase acquosa contenente l'RNA totale}
\begin{itemize}
  \item Attendere 15min evitando di movimentare il campione per un ordinamento parziale e visibile delle fasi
  \item Centrifugare 15min 12000 giri a 4°C per ottenere una separazione netta
\end{itemize}
\end{noSplitBlock}

\begin{noSplitBlock}
\item \textbf{Separazione dell'RNA e precipitazione con 2-propanolo}

{\footnotesize \textbf{Obiettivo}: Precipitare l'RNA dalla fase acquosa ottenendo un pellet visibile.}

\begin{itemize}
  \item Trasferire la fase acquosa in una nuova provetta.
  \item Aggiungere 0.5~ml di 2-propanolo (1:2 con TRIzol) e agitare delicatamente.
\end{itemize}

\begin{percheBox}
\textbf{Perché:} Il 2-propanolo favorisce le interazioni elettrostatiche tra i gruppi fosfato dell’RNA (negativi) e i sali positivi in soluzione. Questa neutralizzazione delle cariche rende l’RNA meno solubile, provocandone la precipitazione come pellet visibile dopo la centrifugazione.
\end{percheBox}

\begin{percheBox}
\textbf{Perché:} Sia 2-propanolo che etanolo fanno precipitare gli acidi nucleici riducendone la solubilità in acqua. Il 2-propanolo, grazie alla sua polarità leggermente inferiore e alla struttura più idrofoba, è più “aggressivo” perché interagisce meno con l’acqua e forza la precipitazione dell’RNA in modo più rapido ed efficace. L’etanolo è più delicato e adatto per i lavaggi finali.
\end{percheBox}
\end{noSplitBlock}

\begin{noSplitBlock}
\item \textbf{Attesa al freddo e centrifugazione finale}

{\footnotesize \textbf{Obiettivo}: Ottenere un pellet compatto di RNA attraverso la precipitazione e la successiva centrifugazione.}

\begin{itemize}
  \item Lasciare il campione a -20~$^\circ$C per 30 minuti.
  \item Centrifugare a 12.000~g per 15 minuti a 4~$^\circ$C.
\end{itemize}
\end{noSplitBlock}











\begin{noSplitBlock}
\item \textbf{Lavaggio con etanolo 70\%}

{\footnotesize \textbf{Obiettivo}: Rimuovere sali e impurità residue dal pellet di RNA, migliorandone la purezza.}

\begin{itemize}
  \item Rimuovere il surnatante e aggiungere 1~ml di etanolo 70\% per ogni ml di TRIZOL usato.
  \item Vortexare brevemente il campione.
\end{itemize}

\begin{percheBox}
\textbf{Perché:} L’etanolo 70\% reidrata il pellet (riespetto al 2-propanolo) e rimuove i contaminanti idrosolubili, lasciando l’RNA più pulito per le successive analisi.
\end{percheBox}
\end{noSplitBlock}


\begin{noSplitBlock}
\item \textbf{Centrifugazione finale}

{\footnotesize \textbf{Obiettivo}: Raccogliere l’RNA in un pellet compatto dopo il lavaggio con etanolo.}

\begin{itemize}
  \item Centrifugare a 12.000~g per 5 minuti a 4~$^\circ$C.
\end{itemize}

\begin{accorgimentoBox}
\textbf{Accorgimento:} Mantenere la temperatura bassa per preservare l’integrità dell’RNA e minimizzare l’attività delle RNasica.
\end{accorgimentoBox}
\end{noSplitBlock}

\newpage
\begin{noSplitBlock}
\item \textbf{Asciugatura del pellet}

{\footnotesize \textbf{Obiettivo}: Eliminare i residui di etanolo senza lasciare asciugare eccessivamente l’RNA, preservandone la stabilità.}

\begin{itemize}
  \item Lasciare il pellet all’aria per circa 10 minuti, evitando di asciugarlo completamente.
\end{itemize}

\begin{criticitaBox}
\textbf{Criticità:} Se l’etanolo non viene rimosso completamente, può compromettere la qualità dell’RNA. Tuttavia, un’asciugatura eccessiva rischia di danneggiare o degradare l’RNA.
\end{criticitaBox}
\end{noSplitBlock}


\begin{noSplitBlock}
\item \textbf{Risospensione finale}

{\footnotesize \textbf{Obiettivo}: Solubilizzare l’RNA in un ambiente privo di nucleasi per conservarlo in modo sicuro e stabile.}

\begin{itemize}
  \item Aggiungere circa 100~$\mu$l di acqua DEPC al pellet.
  \item Risospendere accuratamente il pellet mediante pipettaggio o agitazione delicata.
\end{itemize}

\begin{percheBox}
\textbf{Perché:} L’acqua DEPC è priva di RNasi e protegge l’RNA da eventuale degradazione enzimatica, garantendone la stabilità per le successive analisi.
\end{percheBox}
\end{noSplitBlock}



\end{enumerate}

\subsection{Conclusioni}

Il protocollo di estrazione dell’RNA con TRIZOL ha permesso di isolare RNA totale di buona qualità e quantità dalle cellule vegetali. I passaggi fondamentali — come la lisi in ambiente acido, la separazione delle fasi e la successiva precipitazione con 2-propanolo — hanno garantito l’eliminazione delle proteine e del DNA genomico, arricchendo la frazione acquosa di RNA puro. I lavaggi con etanolo e l’utilizzo di acqua DEPC hanno ulteriormente migliorato la purezza e la stabilità dell’RNA, minimizzando il rischio di degradazione.
\newpage