\section {Spettrometria di un campione di RNA totale}

\subsection{Introduzione}

In questa esperienza, viene analizzato il campione di RNA totale estratto da cellule vegetali mediante il metodo TRIzol. La spettrometria rappresenta un passaggio fondamentale per valutare la qualità e la concentrazione dell’RNA isolato, informazioni indispensabili per eventuali applicazioni successive, come la retrotrascrizione o la PCR.
\newline

Due metodi di spettrofotometria vengono utilizzati:

\begin{itemize}
  \item {Spettrofotometria a cuvetta}
  \item {Spettrofotometria nanodrop}
\end{itemize}

\subsection{Analogie e diversità dei due strumenti}
Entrambi gli strumenti forniscono un’analisi quantitativa e qualitativa dell’RNA. Lo spettrofotometro a cuvetta utilizza un volume maggiore (normalmente 1~ml) e può essere più preciso per campioni diluiti o con contaminanti. È più indicato per campioni con abbondanza di RNA e per misurazioni tradizionali. Il nanodrop, invece, richiede solo 1-2~$\mu$l di campione, è più rapido e pratico, e consente misurazioni su campioni limitati. Tuttavia, la sua accuratezza può essere leggermente inferiore per campioni molto sporchi o molto diluiti.

\subsection{Interpretazione dei dati}

Lo spettrofotometro a cuvetta fornisce i valori di assorbanza a diverse lunghezze d’onda, in particolare:

\begin{itemize}
  \item \textbf{A\textsubscript{260}}: misura l’assorbanza dell’RNA. Viene utilizzata per calcolare la concentrazione applicando la legge di Lambert-Beer\footnote{La legge di Lambert-Beer afferma che l’assorbanza (A) è proporzionale alla concentrazione (c) e alla lunghezza del cammino ottico (l), secondo la relazione: $A = \varepsilon \cdot c \cdot l$.}.

  \item \textbf{A\textsubscript{280}}: misura la presenza di proteine contaminanti, poiché le proteine assorbono fortemente a questa lunghezza d’onda.
  \item \textbf{A\textsubscript{230}}: misura la presenza di composti organici e sali (come fenolo e guanidina).
\end{itemize}

\textbf{Rapporti chiave per la purezza:}
\begin{itemize}
  \item \textbf{A\textsubscript{260}/A\textsubscript{280} $\approx$ 2.0}: indica RNA puro, privo di contaminazione proteica.
  \item \textbf{A\textsubscript{260}/A\textsubscript{230} $\geq$ 2.0}: indica assenza significativa di composti organici e sali.
\end{itemize}

Valori inferiori a questi rapporti suggeriscono la presenza di contaminanti e la necessità di ulteriori purificazioni.

\begin{noSplitBlock}
  \subsection{Registrazione del bianco}

Prima di procedere alla misurazione dei campioni, è fondamentale effettuare la registrazione del bianco, detta anche taratura o calibrazione dello strumento. Questa procedura consiste nella misura dell’assorbanza di una cuvetta contenente esclusivamente il solvente o il tampone utilizzato per la risospensione dell’RNA (ad esempio acqua DEPC o tampone TE).
\newline
\newline
\textbf{Scopo:}  
La registrazione del bianco consente di eliminare dall’assorbanza misurata i contributi dovuti al solvente e alla cuvetta stessa, garantendo che i valori ottenuti siano riferiti esclusivamente all’RNA presente nel campione.

\end{noSplitBlock}
\newpage