\section {Preparazione piastre per trasformazione batterica}

\subsection{Introduzione}
Le piastre di LB-agar arricchite con antibiotici e indicatori cromogenici sono fondamentali per la selezione e il riconoscimento dei batteri trasformati con plasmidi contenenti geni di resistenza o reporter come \textit{lacZ}. In questa sezione viene spiegato come preparare due tipologie di piastre: \textbf{Piastra con Kan/Caf} e \textbf{Piastra con Amp/IPTG/Xgal}.

\subsection{Approfondimento su Piastra con Kan/Caf}
La \textbf{Piastra con Kan/Caf} contiene \textbf{kanamicina} e \textbf{cloramfenicolo} come antibiotici selettivi. Questi antibiotici inibiscono la crescita dei batteri che non contengono il plasmide con i geni di resistenza corrispondenti. Le colonie che crescono su queste piastre possiedono un plasmide che conferisce resistenza a entrambi gli antibiotici. Questa strategia è utile per la selezione di cloni trasformati in esperimenti di clonaggio o in espressioni in cui è richiesta la selezione multipla.


\begin{insightBox}
  \textbf{Insight:} La \textbf{kanamicina} è un antibiotico aminoglicosidico che si lega alla subunità 30S del ribosoma batterico, bloccando la sintesi proteica e causando la morte cellulare. Il gene di resistenza \textit{kanR} codifica un enzima che inattiva la kanamicina.
\end{insightBox}

\begin{insightBox}
  \textbf{Insight:} Il \textbf{cloramfenicolo} agisce legandosi alla subunità 50S del ribosoma batterico, inibendo la formazione del legame peptidico durante la traduzione. La resistenza è mediata dall’enzima cloramfenicolo acetiltransferasi (CAT), che acetila e inattiva l’antibiotico.
\end{insightBox}

\subsection{Approfondimento su Piastra con Amp/IPTG/Xgal}
La \textbf{Piastra con Amp/IPTG/Xgal} combina la selezione antibiotica (ampicillina) con un sistema cromogenico. 
\begin{itemize}\footnotesize
  \item \textbf{Ampicillina}: seleziona i batteri trasformati contenenti il plasmide con il gene di resistenza.
  \item \textbf{IPTG}: è un induttore del promotore \textit{lac}, stimolando l’espressione del gene \textit{lacZ} codificante la $\beta$-galattosidasi.
  \item \textbf{X-gal}: substrato cromogenico che, in presenza di $\beta$-galattosidasi, produce un composto blu.
\end{itemize}
Le colonie che esprimono $\beta$-galattosidasi appaiono blu, mentre quelle che contengono un inserto nel gene \textit{lacZ} (interruzione) restano bianche.


\begin{insightBox}
  \textbf{Insight:} L’\textbf{ampicillina} è un antibiotico $\beta$-lattamico che inibisce la sintesi della parete cellulare batterica. Il gene di resistenza \textit{bla} codifica l’enzima $\beta$-lattamasi, che scinde l’anello $\beta$-lattamico dell’ampicillina e la inattiva.
\end{insightBox}

\begin{insightBox}
  \textbf{Insight:} Gli antibiotici $\beta$-lattamici sono una famiglia ampia e fondamentale in medicina e biologia molecolare. 
  Comprendono: 
  \begin{itemize}\footnotesize
    \item \textbf{Penicilline} (es. ampicillina, amoxicillina)
    \item \textbf{Cefalosporine} (diverse generazioni, es. cefalexina, ceftriaxone)
    \item \textbf{Carbapenemi} (es. imipenem, meropenem)
    \item \textbf{Monobattami} (es. aztreonam)
  \end{itemize}
  Tutti questi antibiotici condividono una qualche forma piu o meno elaborata dell’anello $\beta$-lattamico, essenziale per bloccare gli enzimi responsabili della sintesi della parete batterica.
\end{insightBox}

\begin{percheBox}
  \textbf{Perché la selezione multipla?}
\vspace{0.5em}\\
\footnotesize{  La \textbf{selezione multipla} con due antibiotici (es. Kan/Caf) permette di selezionare batteri che contengono plasmidi con entrambi i geni di resistenza, aumentando la \textbf{stringenza della selezione} e riducendo i falsi positivi.
  \vspace{0.5em}\\Questa strategia è utile anche per: }
  \begin{itemize}
    \item Garantire che l’intero plasmide (con più cassette geniche) sia mantenuto intatto.
\item Plasmidi \textbf{shuttle}, progettati per replicarsi e funzionare in due organismi diversi (es. batteri e lieviti) grazie a origini di replicazione e geni di resistenza specifici per ciascun ospite.

  \end{itemize}
\end{percheBox}

\subsection{Obiettivo}
Preparare piastre LB-agar supplementate con antibiotici e substrati cromogenici per la successiva trasformazione e selezione dei batteri DH5$\alpha$.

  \vspace{1em}
\twoColumnLayout
    {Reagenti:}
    {
      \item Bilancia analitica
      \item Becher
      \item Bacchetta di vetro o magnete per agitazione
      \item Microonde
      \item Falcon
      \item Piastre Petri
      \item Pipette e puntali
      \item Cappa a flusso laminare
    }
  {Strumentazione:}
  {
  \item LB-agar solido
  \item Soluzioni stock di antibiotici:
    \begin{itemize}\footnotesize
      \item Kanamicina 1000X
      \item Cloramfenicolo (CAF) 1000X
      \item Ampicillina 1000X
    \end{itemize}
  \item Soluzioni di:
    \begin{itemize}\footnotesize
      \item IPTG 0.5 mM
      \item X-gal 80 $\mu$g/mL
    \end{itemize}
  \item Acqua milliQ
  }

\subsection{Procedura}
\begin{enumerate}\footnotesize
  \item Sciogliere il terreno LB-agar (se già solidificato) nel microonde o su piastra riscaldante.
  \item Lasciare raffreddare leggermente e, sotto cappa, trasferire 25 mL in una Falcon sterile.
  \item Aggiungere 25 $\mu$L di CAF 1000X e 25 $\mu$L di Kan 1000X. Mescolare e versare in una piastra Petri (\textbf{Piastra con Kan/Caf}).
  \item Ai rimanenti 25 mL aggiungere:
    \begin{itemize}
      \item 25 $\mu$L di Ampicillina (dallo stock 1000X)
      \item 25 $\mu$L di IPTG 0.5 mM
      \item 25 $\mu$L di X-gal 80 $\mu$g/mL
    \end{itemize}
    Mescolare bene e versare in un’altra piastra Petri (\textbf{Piastra con Amp/IPTG/Xgal}).
  \item Lasciare le piastre semi-aperte sotto cappa fino a completa solidificazione dell’agar.
  \item Chiudere e conservare le piastre a 4~$^\circ$C.
\end{enumerate}

\subsection{Conclusioni}
Le piastre preparate con antibiotici e substrati cromogenici sono pronte per la selezione dei batteri trasformati. La \textbf{Piastra con Kan/Caf} permette la selezione di plasmidi con geni di resistenza alla kanamicina o cloramfenicolo, mentre la \textbf{Piastra con Amp/IPTG/Xgal} consente sia la selezione con ampicillina sia la rivelazione di colonie “blue/white” per il gene \textit{lacZ}. Questa preparazione garantisce un ambiente selettivo e cromogenico essenziale per esperimenti di clonaggio e trasformazione.

\newpage
